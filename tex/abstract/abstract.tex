\documentclass{abstract}
\usepackage{amssymb}
\usepackage{amsmath}
\usepackage[lBrack,rBrack,llparenthesis,rrparenthesis]{stmaryrd} % for comparison with 4 similar parenthesis symbols
\usepackage{caption}
\usepackage{pgf}
\usepackage{tikz-cd}
\usepackage{mathtools}
\usepackage[utf8]{inputenc}
\usepackage{listingsutf8}

\begin{document}
\begin{theses}{english}

\title{(Co-)Inductive Dependent Types Normal Form Representation in Pure Type System, the Calculus of Constructions}
\author[1]{Pavlo Maslyanko}{mpp@ntu-kpi.kiev.ua}
\author[2]{Maxim Sokhatsky}{maxim@groupoid.co}
\author[3]{Pavlo Lyutko}{liutko@groupoid.co}
\organization[1]{Faculty of Applied Mathematics at NTUU ``KPI'', Kyiv, Ukraine}
\organization[2]{Synrc Research Center, s.r.o, Praha, Czech Republic}
\organization[3]{Groupoid Infinity, Inc.}
\maketitle

\begin{abstract}
While modern ITT provers expose core axioms of inductive constructions at checker-level,
we promote pure type system setting for inductive constructions, and we show the
technique of Berrarducci-based encoding with an example of Setoid base object along with its theorems.
By preserving cathegorical semantics we unveil the universal method of modeling the
inductive construction as initial object in categories of fibrational F,G-dialgebras (or F-algebra
for non-dependent case) with an example of compilation to Erlang untyped lambda calculus.
\end{abstract}

\paragraph{Motivation}

In modern ITT theorem provers (Lean, Coq, Agda, Idris) you may find inside typechecker core more that
single axiom of *0 : *1, such as fixpoint axiom, axioms for inductive definitions or recursors.
We are trying to avoid this practice and encode execution stream within the pure calculus.
The universes extension axiom is more general, states that *[n] : *[n+1]. This only axiom and four
rules of lambda with dependent types are taken as core language. Thus no recursion,
no effects (they will be modelled with inductive free monad type), only recursive type definitions are allowed.
Having small core typechecker we can reason on it more easily; disabling recursion gives
us full term normalization; small provable compiler core unveils the door to the joyful metacircularity.

By simplifying and reducing the target language we simplify one to one compilation to untyped
lambda languages such as JavaScript, Erlang and others. Moreover the physical representation
of FPGA elements allows here structure preserving transformations too. We use this technique to prototype
 Martin-Löf type system in Erlang along with its language Exe and its prelude.
The system is supposed in future to be capable with modeling of HoTT by providing infinity-groupoids in the base library.

This article will show your some details of Exe prelude and give you full guide on path of inductive
List compilation along with its categorical model.

\paragraph{Lambda Assembler}

Intermediate language resemble both the Henk\cite{Henk} theory of PTS CoC language
and can be tooled with its Morte\cite{morte} implementation:

\begin{lstlisting}[mathescape=true]
    E := * | $\Box$ | ( E ) | E E 
               | $\lambda$ ( L : E ) $\rightarrow$ E 
               | $\forall$ ( L : E ) $\rightarrow$ E
\end{lstlisting}

\paragraph{Exe Language}

We extend the core PTS language with inductive {\bf data} and coinductive {\bf record} definitions.
This constructions is used to model any type in the universe hierarchy. Top level language supports
only (co-)inductive definitions that compiles directly to CoC lambda assembler.

\vspace{0.5cm}
\begin{lstlisting}[mathescape=true]
     I := #identifier
     O := $\varnothing$ | ( O ) |
          $\Box$ | $\forall$ ( I : O ) $\rightarrow$ O |
          * | $\lambda$ ( I : O ) $\rightarrow$ O |
          I | O $\rightarrow$ O | O O
     L := $\varnothing$ | L I
     A := $\varnothing$ | A ( L : O ) | A O
     F := $\varnothing$ | F ( I : O ) | ()
     P := I O , P | I O
     E := $\varnothing$ | E data I : A := F
            | E record I : A [ extend P ] := F
\end{lstlisting}

   \paragraph{Algebras}

   F-Algebras gives us a categorical understanding recursive types.
   Let $F : C \rightarrow C$ be an endofunctor on category $C$.
   An F-algebra is a pair $(C, \phi)$, where C is an object and $\phi\ : F\ C \rightarrow C$
   an arrow in the category C. The object C is the carrier and the functor
   F is the signature of the algebra. Reversing arrows gives us F-Coalgebra.

\begin{center}
\begin{tabular}{lcl}
\begin{tikzcd}
  F\ C \arrow{d}[left]{F\ f} \arrow{r}{\varphi} & C \arrow{d}{f} \\
  F\ D \arrow{r}{\psi} & D \end{tikzcd} & & \begin{tikzcd}
  C \arrow{d}[left]{f} \arrow{r}{\varphi} & F\ C \arrow{d}{F\ f} \\
  D \arrow{r}{\psi} & F\ D \end{tikzcd} \\
  \ & \  &\  \\
  $f \circ \varphi = \psi \circ F\ f$ & & $\psi \circ f =  F\ f \circ \varphi$ \\
\end{tabular}
\end{center}

  \paragraph{Initial Algebras}

  A F-algebra $(\mu F, in)$ is the initial F-algebra if for any F-algebra $(C, \varphi)$
  there exists a unique arrow $\llparenthesis \varphi \rrparenthesis : \mu F \rightarrow C$ where $f = \llparenthesis \varphi \rrparenthesis$
  and is called catamorphism. Similar a F-coalgebra $(\nu F, out)$ is the terminal
  F-coalgebra if for any F-coalgebra $(C, \varphi)$ there exists unique arrow
  $\llbracket \varphi \rrbracket : C \rightarrow \nu F$ where $f = 
  \llbracket \varphi \rrbracket$

\begin{center}
\begin{tabular}{lcl}
\begin{tikzcd}
  F\ \mu F \arrow{d}[left]{F\ \llparenthesis \varphi \rrparenthesis} \arrow{r}{in} & \mu F \arrow{d}{\llparenthesis \varphi \rrparenthesis} \\
  F C \arrow{r}{\varphi} & C \end{tikzcd} & & \begin{tikzcd}
  C \arrow{d}[left]{ \llbracket \varphi \rrbracket} \arrow{r}{\phi} & F\ C\arrow{d}{F\ \llbracket \varphi \rrbracket} \\
  \nu F \arrow{r}{out} & F \nu F\end{tikzcd} \\
  \ & \  &\  \\
  $f \circ in = \varphi \circ F\ f \equiv f = \llparenthesis \varphi \rrparenthesis$& &
  $out \circ f = F\ f \circ \varphi \equiv f = \llbracket \varphi \rrbracket$ \\
\end{tabular}
\end{center}

  \paragraph{Inductive Types}
  There is two types of recursion: one is least fixed point (as $F_A\ X = 1 + A\times X$ or $F_A\ X = A + X\times X$),
  in other words the recursion with a base (terminated with a bounded value), lists are trees are
  examples of such recursive structures (so we call induction recursive sums); and the second
  is greatest fixed point or recursion withour base (as $F_A\ X = A\times X $) --- such kind of
  recursion on infinite lists (codata, streams, coinductive types) we can call recursive products.

\begin{center}
  Natural Numbers: $\mu\ X \rightarrow 1 + X$ \\
  List A: $\mu\ X \rightarrow 1 + A \times X$ \\
  Lambda calculus: $\mu\ X \rightarrow 1 + X \times X + X$ \\
  Stream: $\nu\ X \rightarrow A \times X$ \\
  Potentialy Infinite List A: $\nu\ X \rightarrow 1 + A \times X$ \\
  Finite Tree: $\mu\ X \rightarrow \mu\ Y \rightarrow 1 + X \times Y = \mu\ X = List\ X$ \\
\end{center}

  \paragraph{Lists}
  The data type of lists over a given set A can be represented as the initial algebra
  $(\mu L_A, in)$ of the functor $L_A(X) = 1 + (A \times X)$. Denote $\mu L_A = List(A)$.
  The constructor functions $nil: 1 \rightarrow List(A)$ and
  $cons: A \times List(A) \rightarrow List(A)$ are defined by
  $nil = in \circ inl$ and $cons = in \circ inr$, so $in = [nil,cons]$.
  Given any two functions $c: 1 \rightarrow C$ and $h: A \times C \rightarrow C$,
  the catamorphism $f = \llparenthesis [c,h] \rrparenthesis : List(A) \rightarrow C$
  is the unique solution of the equation system:

$$
\begin{cases}
  f \circ nil  = c \\
  f \circ cons = h \circ (id \times f)
\end{cases}
$$

  where $f = foldr(c,h)$. Having this the initial algebra is presented with functor
  $\mu (1 + A \times X)$ and morphisms sum $[1 \rightarrow List(A), A \times List(A) \rightarrow List(A)]$
  as catamorphism. Using this encoding the base library of List will have following form:

$$
\begin{cases}
 foldr = \llparenthesis [ f \circ nil , h] \rrparenthesis, f \circ cons = h \circ (id \times f)\\
 len = \llparenthesis [ zero, \lambda\ a\ n \rightarrow succ\ n ] \rrparenthesis \\
 (++) = \lambda\ xs\ ys \rightarrow \llparenthesis [ \lambda (x) \rightarrow ys, cons ] \rrparenthesis (xs) \\
 map = \lambda\ f \rightarrow \llparenthesis [ nil, cons \circ (f \times id)] \rrparenthesis
\end{cases}
$$

  \paragraph{Lists in Exe language}
  
  We encode List as usually we do in $\Pi,\Sigma$-provers.

\begin{lstlisting}[mathescape=true]
             data list: (A: *) $\rightarrow$ * :=
                  (nil: list A)
                  (cons: A $\rightarrow$ list A $\rightarrow$ list A)
\end{lstlisting}

$$
\begin{cases}
list = \lambda\ ctor \rightarrow \lambda\ cons \rightarrow \lambda\ nil \rightarrow ctor\\
cons = \lambda\ x\ \rightarrow \lambda\ xs \rightarrow \lambda\ list \rightarrow \lambda\ cons \rightarrow\ \lambda\ nil \rightarrow cons\ x\ (xs\ list\ cons\ nil)\\
nil = \lambda\ list \rightarrow \lambda\ cons \rightarrow \lambda\ nil \rightarrow nil\\
\end{cases}
$$

\begin{lstlisting}[mathescape=true]
           record lists: (A B: *) :=
                  (len: list A $\rightarrow$ integer)
                  ((++): list A $\rightarrow$ list A $\rightarrow$ list A)
                  (map: (A $\rightarrow$ B) $\rightarrow$ (list A $\rightarrow$ list B))
                  (filter: (A $\rightarrow$ bool) $\rightarrow$ (list A $\rightarrow$ list A))
\end{lstlisting}
$$
\begin{cases}
len = foldr\ (\lambda\ x\ n \rightarrow succ\ n)\ 0\\
(++) = \lambda\ ys \rightarrow foldr\ cons\ ys\\
map = \lambda\ f \rightarrow foldr\ (\lambda x\ xs \rightarrow cons\ (f\ x)\ xs)\ nil\\
filter = \lambda\ p \rightarrow foldr\ (\lambda x\ xs \rightarrow if\ p\ x\ then\ cons\ x\ xs\ else\ xs)\ nil\\
foldl = \lambda\ f\ v\ xs = foldr\ (\lambda\ xg\rightarrow\ (\lambda \rightarrow g\ (f\ a\ x)))\ id\ xs\ v\\
\end{cases}
$$

\paragraph{Lambda Assembler}

One List/Type constructor and two List constructors Cons and Nil.

%\begin{lstlisting}[mathescape=true]

\paragraph{List}: $\lambda$ (a: *)
$\rightarrow$ $\forall$ (List: *)
$\rightarrow$ $\forall$ (Cons:
    $\forall$ (head: a)
  $\rightarrow$ $\forall$ (tail: List)
  $\rightarrow$ List)
$\rightarrow$ $\forall$ (Nil: List)
$\rightarrow$ List\\

\paragraph{Cons}: $\lambda$ (a: *)
$\rightarrow$ $\lambda$ (head: a)
$\rightarrow$ $\lambda$ (tail:
    $\forall$ (List: *)
  $\rightarrow$ $\forall$ (Cons:
      $\forall$ (head: a)
    $\rightarrow$ $\forall$ (tail: List)
    $\rightarrow$ List)
  $\rightarrow$ $\forall$ (Nil: List)
  $\rightarrow$ List)
$\rightarrow$ $\lambda$ (List: *)
$\rightarrow$ $\lambda$ (Cons:
    $\forall$ (head: a)
  $\rightarrow$ $\forall$ (tail: List)
  $\rightarrow$ List)
$\rightarrow$ $\lambda$ (Nil: List)
$\rightarrow$ Cons head (tail List Cons Nil)\\

\paragraph{Nil}: $\lambda$ (a: *)
$\rightarrow$ $\lambda$ (List: *)
$\rightarrow$ $\lambda$ (Cons:
    $\forall$ (head: a)
  $\rightarrow$ $\forall$ (tail: List)
    $\rightarrow$ List)
$\rightarrow$ $\lambda$ (Nil: List)
$\rightarrow$ Nil

%\end{lstlisting}

\paragraph{Erlang}

Result Language.

\begin{lstlisting}[mathescape=true]

 ap(Fun,Args) -> lists:foldl(fun(X,Acc) -> Acc(X) end,Fun,Args).

 list () -> fun (List) -> fun (Cons) -> fun (Nil) -> List end end end.
 nil  () -> fun (Cons) -> fun (Nil) -> Nil end end.
 cons () -> fun (A) -> fun (List) -> fun (Cons) -> fun (Nil) -> 
            ap(Cons,[A,ap(List,[Cons,Nil])]) end end end end.
\end{lstlisting}

\end{theses}
\end{document}
